\section{Mineração de Dados Educacionais}
\label{sec:edm}

Sistemas informatizados possuem a capacidade de armazenar em detalhes as interações entre sistema e usuário. Com isto, coleções de dados educacionais, assim como em diversas outras áreas, tomam proporções grandiosas. Ocorre então a preocupação com a criação e aplicação de métodos que detectam padrões nestas bases, que poderiam passar despercebidos devido ao volume das mesmas.

A mineração de dados aplicada à educação (\textit{educational data mining} - EDM) é um campo de pesquisa interdisciplinar e possui como foco o desenvolvimento de métodos para exploração de dados oriundos de ambientes educacionais. Seu objetivo é compreender como estudantes aprendem, identificando as condições em que ocorre este aprendizado, para melhorar resultados e explicar fenômenos educacionais \cite{romero2013data}.

De acordo com \citeonline{scheuer2012educational}, a EDM agrega métodos e ferramentas do amplo campo de mineração de dados; inteligência artificial, na distinção de atributos chave para EDM e abordagem dos problemas teóricos e práticos envolvidos; bem como aprendizado de máquina, utilizando mineração de textos, análise de registros de internet (\textit{web log}) e psicometria para compreender a linguagem natural, perfis de busca e as habilidades que o estudante possui.

Os diversos métodos para lidar com mineração de dados educacionais presentes na literatura podem ser categorizados por \citeonline{baker2010data} da seguinte maneira:

\begin{enumerate}[label=\roman*.]
    \item Predição {--} Utilizada em pesquisas que buscam predizer resultados educacionais sem que seja necessário conhecer fatores intermediários, além de inferir qual seria o valor de saída em contextos que não é desejável obter um rótulo direto (\textit{e.g.} estudos em que a obtenção de rótulos alteraria o comportamento a ser analisado);
    \item Análise de agrupamento {--} Aplicada à diferentes níveis de detalhamento (granularidade) a fim de investigar as diferenças entre escolas, estudantes individuais ou mesmo as ações destes estudantes;
    \item Mineração de relacionamento {--} Em EDM são observadas as ações, comportamentos de aprendizagem e decisões eventuais dos estudantes;
    \item Descoberta com modelos {--} Um modelo desenvolvido por meio de predição, análise de agrupamentos ou, em alguns casos, por meios humanos é utilizado como componente em outras análises, como mineração de relacionamentos ou predição;
    \item Destilação de dados para julgamento humano {--} Técnicas de visualização da informação são utilizadas para facilitar a compreensão humana e reconhecimento de padrões (\textit{e.g.} curva de aprendizado).
\end{enumerate}

Ao longo da literatura, os objetivos da mineração de dados educacionais são descritos de diversas maneiras: de maneira que suportem as reflexões e melhorem a performance de aprendizado dos estudantes, bem como permitam que os professores compreendam os processos de aprendizagem dos alunos. \citeonline{bousbia2014contribution} distinguem os objetivos da EDM em tópicos gerais que incluem:

\begin{enumerate}[label=\roman*.]
    \item Modelagem estudantil {--} Incorpora informações detalhadas sobre o estudante, tais como: nível de conhecimento, habilidades, satisfação, atitudes, experiências e medidas de aprendizado, além de certos problemas que possam afetar negativamente o aprendizado do mesmo. O objetivo é criar e melhorar o modelo para uso das informações dos alunos, por parte dos professores em sala de aula ou em ações da instituição responsável;
    \item Predição de performance estudantil {--} Tem o objetivo de predizer as notas finais dos estudantes ou outras medidas de aprendizado (nível de retenção ou habilidade de aprendizado futuro);
    \item Comunicação para partes interessadas (\textit{stakeholders}) {--} Ajuda administradores e educadores a analisar atividades dos estudantes, bem como o uso de informações nos cursos;
    \item Análise da estrutura de domínio {--} Utiliza a predição da performance estudantil como medida qualitativa na criação de modelos de estrutura de domínio. estes modelos caracterizam o conteúdo a ser aprendido e sequências de instruções otimizadas.
\end{enumerate}

Em um estudo de caso proposto pelos autores em \citeonline{merceron2005educational}, onde algoritmos de mineração são utilizados para a descoberta de padrões em dados obtidos através de uma ferramenta de tutoria baseada na web oferecida na Universidade de Sydney, pôde ser observado que a aplicação de \textit{data mining} produz muitos benefícios. O uso de técnicas de classificação e análise de agrupamentos identificou estudantes que corriam risco de reprovação e padrões de comportamento entre estes alunos, possibilitando a criação de políticas pedagógicas à serem implementadas pela instituição.