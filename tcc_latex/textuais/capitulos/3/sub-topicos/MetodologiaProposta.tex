\section{Metodologia}
\label{sec:metodologia-proposta}

A base de dados utilizada será obtida através da aplicação de jogos sérios como parte do projeto de pesquisa mencionado na Seção \ref{sec:delimitacao-estudo}. Este projeto busca utilizar comitês de classificadores para auxiliar o pré\hyp{}diagnóstico de transtornos de aprendizagem e coleta dados numéricos pertinentes a análise da capacidade cognitiva de crianças com idade entre 7 e 12 anos.

Inicialmente, será feita uma análise da base de dados para que seja identificada a presença e natureza de possíveis inconsistências. Então, de acordo com a necessidade, serão aplicadas as técnicas apresentadas na Seção \ref{sec:data-preproces} (Pré\hyp{}processamento de dados), estas técnicas compreendem: i) correção de valores faltosos (Seção \ref{subsec:data-clean}); ii) suavização de ruídos (Seção \ref{subsubsec:noise}); e iii) detecção de \textit{outliers} (Seção \ref{subsec:outliers}). Também serão aplicadas técnicas de redução de dimensionalidade (Seção \ref{subsec:data-reduction})), que consistem em: i) seleção de atributos (Seção \ref{subsubsec:atribute-selection}); e ii) seleção de instâncias (Seção \ref{subsubsec:instance-selection}), a fim de reduzir a complexidade na execução do classificador.

A aplicação das técnicas de correção necessárias será conduzida com o auxílio da linguagem R, escolhida por ser um software livre (\textit{open source}), que vêm apresentado crescimento ao longo dos anos e fornece diversas ferramentas e bibliotecas para facilitar análises estatísticas.

Por fim, com o objetivo de validar a utilização das técnicas supracitadas, serão aplicados algoritmos de classificação sobre os dados originais e aqueles obtidos após a etapa de pré\hyp{}processamento, para que os resultados obtidos possam ser comparados. Espera\hyp{}se que, através deste processo, possa ser comprovada a importância do pré\hyp{}processamento e a melhora que sua aplicação causa na acurácia e desempenho para a descoberta de conhecimento. A avaliação de desempenho será feita através da comparação da acurácia do classificador, portanto, será observada a importância da aplicação do pré\hyp{}processamento caso: $Acur\acute{a}cia \left ( D^{'} \right ) > Acur\acute{a}cia \left ( D \right )$, onde $D$ representa a base de dados antes da etapa de pré\hyp{}processamento e $D^{'}$ após.